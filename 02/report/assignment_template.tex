\documentclass{article} % This command is used to set the type of document you are working on such as an article, book, or presenation

\usepackage{geometry} % This package allows the editing of the page layout
\usepackage{amsmath}  % This package allows the use of a large range of mathematical formula, commands, and symbols
\usepackage{graphicx}  % This package allows the importing of images
\usepackage{tabularx}
\usepackage[square,numbers]{natbib}
\usepackage {hyperref}
\usepackage{fancyhdr}
\pagestyle{fancy}
\usepackage{xcolor}
\usepackage{listings}
\usepackage{appendix}

\setcitestyle{square}
\bibliographystyle{abbrvnat}

\newcommand{\sol}{\textbf{Solution}:} %Use if you want a boldface solution line
\newcommand{\maketitletwo}[2][]{
\begin{center}
        \Large{\textbf{Assignment #1 - #2}
            
            CV701 - Human and Computer Vision} % Name of course here
        \vspace{5pt}
        
        \normalsize{Group Member 1, Group Member 2  % Group Member Names
        
        }      
        \vspace{10pt}       
\end{center}
}

\newcommand{\outlineitem}[2][]{
	#1. #2 \\ \hline 
}

% Header Logo
\setlength\headheight{39pt}
\fancyhead[L]{{\large{CV701}}}
\rhead{\includegraphics[width=5cm]{./images/thumbnail-MBZUAI-Logo}}

\begin{document}

\maketitletwo[1]{Group Name}  % Change the group name here

\section*{Task 1}

\begin{figure}[h]
    \centering
    \includegraphics[]{./images/thumbnail-MBZUAI-Logo.png}
    \caption{First Photo Captured.}
    \label{fig:photo1}
\end{figure}

An example citation \cite{Misra2020Jan}.

\section*{Task 2}

\subsection*{Task 2.1}
The code to convert RGB color representation from RGB to HSI can be found in Appendix \ref{app:rgb_to_hsi}.


\subsection*{Task 2.2}

\bibliography{references}

\appendix

\section*{Appendices}
\section{ Code for RGB to HSI conversion}
\label{app:rgb_to_hsi}

\begin{lstlisting}
def rgb_2_hsi(rgb_img: np.ndarray) -> np.ndarray:
    r_channel = rgb_img[:,:,0]
    g_channel = rgb_img[:,:,1]
    b_channel = rgb_img[:,:,2]
    r_channel = r_channel.astype(np.float32)

    # Intensity
    intensity = (r_channel + g_channel + b_channel)/3

    # Saturation
    minimum = np.minimum(np.minimum(r_channel, g_channel), b_channel)
    sat = 1-(3/(r_channel + g_channel + b_channel + 0.000001)*minimum)

    # Hue
    hue = np.arccos((0.5 * ((r_channel- g_channel)+(r_channel-b_channel))) / (
        np.sqrt((r_channel-g_channel)**2+(r_channel-b_channel)*(g_channel-b_channel))))

    hue[b_channel>g_channel] = 2*math.pi - hue[b_channel>g_channel]
\end{lstlisting}

\end{document}